\documentclass[a4paper,french,english,10pt]{article}
\usepackage{a4wide}
\usepackage[T1]{fontenc}
\usepackage[latin9]{inputenc}
%\usepackage{geometry}
%\geometry{verbose,a4paper,tmargin=3cm,bmargin=3cm,lmargin=3cm,rmargin=3cm,headheight=3cm,headsep=3cm}
\usepackage{babel}
\usepackage{xcolor}
\usepackage{amsmath}
\usepackage{graphicx}
\usepackage{graphics}
\usepackage{epsfig}
%\usepackage{esint}
\usepackage{color}
\usepackage{amsfonts}
\usepackage{amssymb,latexsym}
\usepackage{amscd}
\usepackage{multirow}
\usepackage{amsthm}
\usepackage[pdftex,bookmarks=true,bookmarksopen=true,colorlinks=true,linkbordercolor=white, citecolor=blue, linkcolor=red]{hyperref}

\usepackage{comment} 


\newcommand{\dr}{\partial_R}
\newcommand{\ds}{\displaystyle}
\newcommand{\dz}{\partial_Z}
\newcommand{\dphi}{\partial_{\phi}}
\newcommand{\ephi}{\mathbf{e}_{\phi}}
\newcommand{\dt}{\partial_t}
\newcommand{\lp}{\triangle_{pol}}
\newcommand{\lpp}{\triangle_{pol}^2}
\newcommand{\gradp}{\nabla_{pol}}
\newcommand{\lpm}{\triangle_{pol}^{-1}}
\newcommand{\gs}{\triangle^{*}}
\newcommand{\U}{\mathbf{U}}
\newcommand{\V}{\mathbf{V}}
\newcommand{\vv}{\mathbf{v}}
\newcommand{\vB}{\mathbf{B}}
\newcommand{\vJ}{\mathbf{J}}
\newcommand{\vE}{\mathbf{E}}
\newcommand{\rot}{\nabla \times}
\newcommand{\er}{\mathbf{e}_R}
\newcommand{\ez}{\mathbf{e}_Z}
\newcommand{\eps}{\varepsilon}

\renewcommand{\thefootnote}{\fnsymbol{footnote}}
\begin{document}

\title{Answers for the Reviewers  for the paper "Proof of uniform convergence for a cell-centered AP discretization of the hyperbolic heat equation on general meshes"}


\author{Emmanuel Franck\footnotemark[1] \footnotemark[2],
C. Buet \footnotemark[3], 
B. Despr\'es\footnotemark[4], 
T. Leroy\footnotemark[4] \footnotemark[3]
}
 \footnotetext[1]{INRIA Nancy Grand-Est and IRMA Strasbourg}
 \footnotetext[3]{CEA, DAM, DIF, F-91297 Arpajon Cedex,France}
 \footnotetext[4]{Laboratoire Jacques-Louis Lions,
  Universit\'e Pierre et Marie Curie,
  75252 Paris Cedex 05,
  France}
\footnotetext[2]{Email adress: emmanuel.franck@inria.fr}

\maketitle


\section{Work done}


The authors deeply thank the referees for their reviews.
We understant that our work is judged interesting and worth of publications,
but that the a more detailed of explanation of the estimates and a proof of the fully 
discrete scheme were necessary.
We have done this work, together with a reexamination line after line of all estimates.

Our answer to the all question raised in the reports is below.
In {\bf bold} are the questions/remarks, our answers follow immediately.



\section{Answers for the report 2M}


\begin{itemize}
\item {\bf A massive overuse of "Constant C" and the issue of the estimates' time-dependence \dots}

In order to give a positive answer to the question of the "massive overuse of the constant $C$", we have
completely re-examine all that concerns the constants.
In order to clarify, we have adopted the following strategy:

a)  the four main constants which correspond to the four branches of the AP diagram and the new AP diagram
are denoted as $~_\downarrow C$, $C^\rightarrow$, $C_\downarrow$ and $C_\leftarrow$.
Moreover the dependence with respect to the constant which measures the quality
of the mesh, denoted as $C_\mathcal M$, and with respect to the Gronwall exponentials
is indicated systematically {\bf in dimension $n=1$}.
In 1D, it is now clear that the Gronwall exponentials have zero dependence with respect to $1/\varepsilon$.
So they are perfectly bounded for a given time $T$. There is no exponential of exponential.
 
In dimension two, we do not indicate the extra dependance 
with respect to $C_\mathcal M$ and  the Gronwall-exponentials,
since it would make the reading much heavier
once again, and  it would bring almost zero with respect to the 1D case.

The constant $S$ in section 3.5 is now renamed $E_3$.

b) the "sub-sectioning" tries to be more rigorous. For example 
a new subsection 3.4 now explains the various norms and approximation/interpolation estimates,
subsection 3.5 is split into sub subsections entitled stability, consistency, ..., 
subsection 3.6 is split into a rescaling sub subsection, and so on.

c) The main inequality (one of the four branch) of the AP diagram yields a lemma
which is written at the beginning of the main subsections (3.5, 3.6, 3.7).


d) A new section (asked by the other referred) explains how to use a simple method
(from Despres, Math of Comp 04) to obtained a $\sqrt{\Delta t}$ error term in time, uniformly
with respect to $h$ and $\varepsilon$.
It allows to formulate the  theorem 1.1 immediately in the introduction. 



In any cases, we mention to the referee that a huge  work has been done to fulfill his expections.
It also seems to us more rigorous now.


\begin{itemize}

\item {\bf Page 7, Proposition 2.2: as far as I understand, the proof yields C = T. So there's no need for this first constant. Agree ? Same for Prop. 3.1 ?}

we have corrected the constant in the proposition 2.2 page 7


\item {\bf The former comment may seem anecdotal, but it reveals (at least) 2 interesting things: }

\begin{itemize}
\item 
{\bf 1.
assuming p smooth, the approximation (on continuous solutions) is reliable mainly for $t <<\frac1 \epsilon$ }

\item {\bf 2. all the remaining Propositions (pages 11-14) ask for the Gronwall lemma in their proofs: hence we obtain bounds growing exponentially in time, suggesting that numerical approximations (discrete solutions) are now r \dots} 

\end{itemize}

{\bf Hence I think that the authors shouldn't try to hide this apparent discrepancy under the carpet, because it may be checked numerically rather easily: it suffices to run the code which produced the Figure 6 several times, and scrutinize whether we see a growth in time which is linear, polynomial or exponential. Moreover, this will give valuable information on the sharpness of the error bounds rigorously proven}

Already answered: the  important parameters of the  constants are now fully detailed in dimension 1, and partially
in dimension 2.


\item {\bf Page 8, the estimate (27): the proof is quite easy, so why not displaying it in order to wipe off the C ? It suffices to set up a classical computation, from e.g. DiPerna,
$$
\frac12 \frac d {dt} \int_R p^2 dx + \frac1\sigma \int_R |\partial_x p |^2\leq 0 \Longrightarrow \|  p(t,\cdot) \|_{L^2}\leq  \|  p_0 \|_{L^2}.
$$
which yields, without introducing any C,
$$
\int_0^T \int_R |\partial_x p |^2 dt \leq \frac \sigma 2 \|  p_0 \|_{L^2}^2.
$$
}

Corrected.


\item {\bf Page 11, proof of (32)-(33)} 

\begin{itemize}

\item {\bf 1.  one can find a constant $C \geq 0$ such that the next term $[\dots]$. Is this
C the one of the grid introduced in Hypothesis 2.1 ?}

It is the mesh constant, now explicitly given as $C_\mathcal M$ everywhere.

\item {\bf 2. The last term in (28) $[\dots]$: seems like it's (29) instead }

(28)$\to$(29), corrected.


\item {\bf 3.  using a Cauchy-Schwarz inequality $[\dots]$: in my opinion, one should invoke instead Jensen's inequality. Indeed \dots}


We have change ``the Cauchy-Schwarz inequality'' to ``the Jensen inequality''.

\end{itemize}


\item {\bf Pages 11-12, in Proposition 2.6, the C(T) is an exponential of the time T: it must be indicated and the discrepancy with respect to the linear growth found in Proposition 2.2 should be commented. Moreover, concerning the estimate (35): the statement is correct, except that it would require a little bit of explanations, as it relies on a bootstrap-type argument based on estimate (27), which wasn?t stated really correctly (as already seen).}

Pages 11-12: all the constants have been given.


\item 
{\bf Pages 20-27: the statements and proofs of Props. 3.4 and 3.5 are A MESS!} 

Entirely reshaped as already mentioned above.


\begin{itemize}
\item 

{\bf 1. When studying the different terms in $E'(t)$, please use either itemize or enumerate, and perhaps forward heavy calculations to an 
Appendix in order to lighten stuff,}

Done.

\item {\bf 2. Props. 3.6-8 are indeed "technical Lemmas" and should be stated accordingly (cf. comments on the 1D case of Prop. 2.8). Their proofs might be forwarded too.}




\item {\bf 3. the 1D comment about Jensen's inequality appears to apply here again, bottom of page 26 }

Jensen does not show up anymore.

\end{itemize}


\item {\bf Page 36, 4: perhaps a bibliographic reference for the solution of the telegraph's equation may be useful to the reader. Moreover, there is a conflict of notation between the $\gamma$ used here and the one invoked in the proofs on pages 25-26-27}

Page 36 : The conflict for the notation have been corrected. For the solution there are not reference, we have construct this solution using a separation of the variable. we have added the details.

\item {\bf As said before, there's a question which should be addressed, that is the rate of amplification in time of the errors: the Gronwall?s Lemma produces a time-exponential rate, so it's interesting to check out numerically if this feature is a discrepancy of the analysis, or if it holds practically. Such a point was addressed, in a totally different context of scalar balance laws.


{\it D. Amadori, L. Gosse, Transient $L1$
 error estimates for well-balanced schemes on non-resonant scalar balance laws, J. Differential Equations 255 (2013) 469-502}


} 

\item Last point: in 1D we have explicit all the constant. We obtain time-exponential rate as you expect. Numerically we constat a a time-linear rate, we have added a remark in the numerical part at this subject. 

\item 
Additional modifications: Page 5 we have modified the constant for the 1D regularity mesh;
We have correct the constant in the proposition 2.5 page 8.
\end{itemize}






\end{itemize}


\begin{itemize}


\item Comments and suggestions
\begin{itemize}
\item 
{\bf Pages 2-': with an Introduction that long, it makes sense to insert "sub- sections" entitled "Precisions on AP discretizations" (right after Remark 1.2), and ?Organization of the paper? (right after the proof of Prop 1.3 }

 We have added two subsections : "Precisions on AP discretizations" and "Organisation of the proof"


\item {\bf Page 5: is there a connection between "nodal-based" FV schemes (advocated in the Intro.) and the classical Godunov scheme where 2D Riemann problems are to be solved at each node of the computational grid ? See for instance the elementary case of a non-resonant scalar balance law in,
L. Gosse, A two-dimensional version of the Godunov scheme for scalar balance laws. SIAM J. Numer. Anal. 52 (2014) 626-652.
If so, the sentence in the Introduction "it strongly questions ..." wouldn't be that surprising. Moreover, on page 27, the authors speak about the "incorporation in the approximate nodal Riemann solver", thus reinforcing this vague impression of similarity with 2D Riemann solvers}


We  added 2 sentences about the 2D Riemann problem and the nodal scheme. 
We agree that there is a similarity, but essentially concerning the final formula itself.
The method is rather different.


\item {\bf Page 6: The big footnote is confusing: in my opinion, it would be better to state clearly that, in order to build a discretization able to produce a 2D AP numerical process, the authors choose to renounce to the well- balanced property of the original 1D Gosse-Toscani scheme (except on uniform Cartesian grids) while retaining only its AP character. This ex- plains why the algebraic expressions differ from each other.} 

We have added a sentence to explain why we use this formulation of the scheme and not the classical Gosse-Toscani scheme. 

For the remark about " the authors choose to renounce to the well- balanced property of the original 1D Gosse-Toscani scheme (except on uniform Cartesian grids) while retaining only its AP character" it is not clear that we say that. Indeed the scheme is well balanced for a subset of steady states (see annex of the paper "design of AP scheme ..."): the steady state associated to $u=Cts$ and $p$ linear. This property is perhaps impotent for the uniform convergence. At the end we renounce only to the steady states associated to $\nabla{u}=\nabla\times \phi$ with $\phi$ a potential. 


\item {\bf Page 7, Proof of Prop. 2.2, the notation choice $u^\epsilon=-\frac\sigma \epsilon \partial _x p^\epsilon$  is confusing:
according to (13), a notation like $v^\epsilon$ looks better.}

 We have correct this part introducing $v$ to obtain a not confusing proof.


\item {\bf Page 10, last lines, what is the definition of $\mathbf u^{\eps}$ ? This notation doesn?t
appear previously.}

 We have replaced is $\mathbf{u}^{\eps}$ by $u^{\eps}$



\item {\bf Page 14, the Proposition 2.8 is indeed a (technical) Lemma, and this should be stated accordingly}

We have replaced the proposition 2.8 by the Lemma 2.8



\item {\bf Page 14, end of the 1D section: according to what is displayed numerically in the �4, Figure 6, we see a second-order accuracy
 (in h) for $\epsilon \approx  h^2$. This appears to be an extension in 2D of an observation made in a paper devoted to 
 "TAHO schemes", see especially �7.1.2 in,
D. Aregba, M. Briani, R. Natalini, Time Asymptotic High Order Schemes for Dissipative BGK Hyperbolic Systems, arXiv:1207.6279}


We have added the reference on the paper "Time asymptotic high order ..." at the end of the 2D in the numerical results.

\item {\bf Page 15, Proposition 3.1: same comment than for Proposition 2.2}

We have corrected the proposition 


\item {\bf Page 15, eqn. (45), there is a "(" missing in the first line.}

We have added the missing "(".
\end{itemize}



\item Minor issues and Typos
\begin{itemize}
\item 1: We have replaced $P^0$ by "first order"
\item 2 : we have corrected this sentence 
\item 3 : we have corrected this sentence 
\item 4 : we have corrected this sentence. (17) correspond to the scheme not the smooth data.
\item 5 : we have corrected this sentence. 
\item 6 : we have corrected your two remarks in this page 
\item 7 : we have corrected this sentence. 
\item 8: We have replaced $\gamma$ by $\tau$ in the numeric part.
\item 9: The date for the L. Gosse paper have been modified. for the S. Jin article we have copy the reference written on the website of S. Jin 
\end{itemize}
\end{itemize}



\section{Answers for the report 2C}

Before answering to explain what we have done, we would like to make a general comment.


We still consider that the space discretization is the most important, in the sense
that such multiD method can be used by researchers and engineers only if they are proved
to be convergent, in the usual sense.
That it is converge for a family of smooth enough functions.
The question of the boundary in time
layer plays little role for the application we have in mind.
On the contrary, discontinuity of the coefficients of the equations may generate 
some boundary layers in space. For radiation equation this is called the Milne problem.
This issue is of tremendous importance, but seems to us difficult to address in general, not only for the scheme that we develop.

On the other hand our method is the first which is proved to be convergent: what we think is of real importance
is the fact that usual Finite Volume technique through the edges are not able have a geometrical
obstruction for AP on a general mesh. This is why our method for which fluxes are corner based 
seems a real advance.

Now we answer to 
{\bf  The issue concerning the time
discretization must be addressed correctly.
}.

This issue was not addressed at all because it seems to us that the convergence in space was the real point.
Nevertheless we do agree with the referee.
This is why was have now done the following.

\begin{itemize}
\item the theorem of convergence for a fully explicit scheme is written in the introduction, and proved
in section  4b by the abstract method (the main idea was given in  despres 04, but
the new proof is much more direct). This method easily give a uniform estimate 
between the continuous and fully discrete solutions, so proves the theorem when combined with the previous results.

\item Furthermore we have restructure most of the material. The proof of the naive estimate in 2D 
is in two parts, the technical details being in the appendix is asked by the other referee.
We also give a much more detailed presentation of the constants
$~_\downarrow C$, $C^\rightarrow$, $C_\downarrow$ and $C_\leftarrow$ with the regularity of the initial data
and other constants.

\item We feel the entire work is more complete rigorous with these two modifications.

\end{itemize}

The proposed method has already been applied to to a non linear diffusion equation in  (see references):
{\it C. Buet, B. Despr\'es, E. Franck, \emph{An asymptotic
preserving scheme with the maximum principle for the $M_1$ model on distorted
meshes}, C.R. Acad. Sci., Paris, S\'er. I, Math., Vol 350,  11-12
pp 633-638,  2012.}


The propose references are relevant. They have been added and discussed with the other references in the introduction.

\end{document}