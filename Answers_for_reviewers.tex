\documentclass[a4paper,french,english,10pt]{article}
\usepackage{a4wide}
\usepackage[T1]{fontenc}
\usepackage[latin9]{inputenc}
%\usepackage{geometry}
%\geometry{verbose,a4paper,tmargin=3cm,bmargin=3cm,lmargin=3cm,rmargin=3cm,headheight=3cm,headsep=3cm}
\usepackage{babel}
\usepackage{xcolor}
\usepackage{amsmath}
\usepackage{graphicx}
\usepackage{graphics}
\usepackage{epsfig}
%\usepackage{esint}
\usepackage{color}
\usepackage{amsfonts}
\usepackage{amssymb,latexsym}
\usepackage{amscd}
\usepackage{multirow}
\usepackage{amsthm}
\usepackage[pdftex,bookmarks=true,bookmarksopen=true,colorlinks=true,linkbordercolor=white, citecolor=blue, linkcolor=red]{hyperref}

\usepackage{comment} 


\newcommand{\dr}{\partial_R}
\newcommand{\ds}{\displaystyle}
\newcommand{\dz}{\partial_Z}
\newcommand{\dphi}{\partial_{\phi}}
\newcommand{\ephi}{\mathbf{e}_{\phi}}
\newcommand{\dt}{\partial_t}
\newcommand{\lp}{\triangle_{pol}}
\newcommand{\lpp}{\triangle_{pol}^2}
\newcommand{\gradp}{\nabla_{pol}}
\newcommand{\lpm}{\triangle_{pol}^{-1}}
\newcommand{\gs}{\triangle^{*}}
\newcommand{\U}{\mathbf{U}}
\newcommand{\V}{\mathbf{V}}
\newcommand{\vv}{\mathbf{v}}
\newcommand{\vB}{\mathbf{B}}
\newcommand{\vJ}{\mathbf{J}}
\newcommand{\vE}{\mathbf{E}}
\newcommand{\rot}{\nabla \times}
\newcommand{\er}{\mathbf{e}_R}
\newcommand{\ez}{\mathbf{e}_Z}
\newcommand{\eps}{\varepsilon}

\renewcommand{\thefootnote}{\fnsymbol{footnote}}
\begin{document}

\title{Answers for the Reviewers  for the paper "Proof of uniform convergence for a cell-centered AP discretization of the hyperbolic heat equation on general meshes"}


\author{Emmanuel Franck\footnotemark[1] \footnotemark[2],
C. Buet \footnotemark[3], 
B. Despr\'es\footnotemark[4], 
T. Leroy\footnotemark[4] \footnotemark[3]
}
 \footnotetext[1]{INRIA Nancy Grand-Est and IRMA Strasbourg}
 \footnotetext[3]{CEA, DAM, DIF, F-91297 Arpajon Cedex,France}
 \footnotetext[4]{Laboratoire Jacques-Louis Lions,
  Universit\'e Pierre et Marie Curie,
  75252 Paris Cedex 05,
  France}
\footnotetext[2]{Email adress: emmanuel.franck@inria.fr}

\maketitle


\section{Answers for the report 2M}
\begin{itemize}
\item A massive overuse of ?Constant C? and the issue of the estimates? time-dependence
\begin{itemize}
\item Page 5 we have modified the constant for the 1D regularity mesh
\item We have correct the constant in the proposition 2.5 page 8
\item 1 : we have corrected the constant in the proposition 2.2 page 7
\item 8: in 1D we have explicit all the constant 
\item 3 :
\item 4 :
\item 5 : We have explicit the constant between the equation (32) and the end of the proof of the proposition 2.5
\item 6 :
\item 7 : The conflict for the notation have been corrected. For the solution there are not reference, we have construct this solution using a separation of the variable. we have added the details.
\item 2-8: in 1D we have explicit all the constant. We obtain time-exponential rate as you expect.
\end{itemize}
\begin{itemize}
\item Comments and suggestions
\begin{itemize}
\item 1: We have added two subsections : "Precisions on AP discretizations" and "Organisation of the proof"
\item 2: we have added 2 sentences about the 2D Riemann problem and the nodal scheme. 
\item 3: We have added a sentence to explain why we use this formulation of the scheme and not the classical Gosse-Toscani scheme. For the remark about " the authors choose to renounce to the well- balanced property of the original 1D Gosse-Toscani scheme (except on uniform Cartesian grids) while retaining only its AP character" it is not clear that we say that. Indeed the scheme is well balanced for a subset of steady states (see annex of the paper "design of AP scheme ..."): the steady state associated to $u=Cts$ and $p$ linear. This property is perhaps impotent for the uniform convergence. At the end we renounce only to the steady states associated to $\nabla{u}=\nabla\times \phi$ with $\phi$ a potential. 
\item 4: We have correct this part introducing $v$ to obtain a not confusing proof.
\item 5: We have replaced is $\mathbf{u}^{\eps}$ by $u^{\eps}$
\item 6 : We have replaced the proposition 2.8 by the Lemma 2.8
\item 7 : We have added the reference on the paper "Time asymptotic high order ..." at the end of the 2D in the numerical results.
\item 8 : We have corrected the proposition 
\item 9 : We have added the missing "("
\end{itemize}
\item Minor issues and Typos
\begin{itemize}
\item 1: We have replaced $P^0$ by "first order"
\item 2 : we have corrected this sentence 
\item 3 : we have corrected this sentence 
\item 4 : we have corrected this sentence. (17) correspond to the scheme not the smooth data.
\item 5 : we have corrected this sentence. 
\item 6 : we have corrected your two remarks in this page 
\item 7 : we have corrected this sentence. 
\item 8: We have replaced $\gamma$ by $\tau$ in the numeric part.
\item 9: The date for the L. Gosse paper have been modified. for the S. Jin article we have copy the reference written on the website of S. Jin 
\end{itemize}
\end{itemize}
\end{document}